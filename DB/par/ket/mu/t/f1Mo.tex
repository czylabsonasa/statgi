független sokaságok, nem ismertek a szórások, de feltesszük hogy
\mcode{egyenlőek}. jelölés: $X:$ Mokka, $Y$: Koffe

\Mat{
%   X=8.2, 5.0, 6.8, 6.7, 5.8, 7.3, 6.4, 7.8\us
%   Y=5.1, 4.3, 3.4, 3.7, 6.1, 4.7   \us
   n_X=8,\ n_Y=6\  \alpha=0.05 \us
   H_{0}: \mu_X-\mu_Y \le 0 \us
   H_{1}: \mu_X-\mu_Y > 0 \us
   \overline{X}=6.75,\ \overline{Y}=4.55\us
   s^2_X=1.0857,\ s^2_Y=0.967\us
   s_p=\sqrt{\frac{7\cdot1.0857+5\cdot0.967}{12}}=1.0180\us
   t=\frac{6.75-4.55}{1.0180\cdot\sqrt{\frac{1}{8}+\frac{1}{6}}} = 4.0017\us
   c_f=t_{0.95,\ df=12}=1.782,\nh
   t \ge c_f
}
Tehát adott szinten a minta nem támasztja alá $H_0$-at - elvetjük, vagyis
elfogadható az az állítás hogy a Mokka lassabban oldódik.
