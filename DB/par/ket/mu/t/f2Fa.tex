Az angliai New Dumber golflabdagyárában egy újfajta golflabda borítást fej-
lesztettek ki. A tesztek azt mutatták, hogy ez az új borítás jóval ellenállóbb,
mint a hagyományos. Felmerült azonban a kérdés hogy az új borítás nem
változtatja-e meg az átlagos ütéstávolságot. Ennek eldöntésére \mcode{42} 
labdát próbáltak ki, \mcode{26} hagyományosat és \mcode{16} labdát az újak közül. A labdákat géppel
lőtték ki, elkerülve ezzel az emberi tényező okozta szóródást. A yardban mért
ütéstávolságok összesítő adatait, mely távolságokat mindkét esetben normális
eloszlásúnak tételezzük fel, az alábbi láthatjuk:
\Mat{
\text{Hagyományos:}\kh n=26,\kh \text{átlag}=271.4,\kh s^2=35.58\us
\text{Új:}\kh n=16, \text{átlag}=268.7,\kh s^2=48.47\us
}
\mcode{90}\%-os szinten vizsgáljuk meg, hogy az új borítás 
megváltoztatja-e az átlagos ütéstávolságot!